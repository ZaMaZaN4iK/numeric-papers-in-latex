%!TEX root = ts.tex
\begin{addedblock}

\rSec0[rational.math]{Rational math}

\rSec1[rational.class]{Class \tcode{rational}}

\begin{codeblock}
#include <compare>

class rational {
public:
    constexpr rational() noexcept = default;
    rational(const rational& rat) = default;
    rational(rational&& rat) noexcept = default;
    rational& operator=(const rational& rat) = default;
    rational& operator=(rational&& rat) noexcept = default;
    ~rational() = default;

    explicit rational(float num);
    explicit rational(double num);
    explicit rational(long double num);

    explicit rational(integer num);

    rational(integer num, integer den);

    rational& operator=(integer num);
    rational& assign(integer num, integer den);

    void swap(rational& rhs) noexcept;

    rational normalize() const;

    integer numer() const;
    integer denom() const;

    explicit operator bool() const noexcept;

    rational& negate() noexcept;
    rational& invert() noexcept;

    rational& operator++();
    rational& operator--();

    rational operator++(int);
    rational operator--(int);

    rational& operator+=(const integer& rhs);
    rational& operator-=(const integer& rhs);
    rational& operator*=(const integer& rhs);
    rational& operator/=(const integer& rhs);

    rational& operator+=(const rational& rhs);
    rational& operator-=(const rational& rhs);
    rational& operator*=(const rational& rhs);
    rational& operator/=(const rational& rhs);

private:
    integer numerator_;   // \expos
    integer denominator_; // \expos
};
\end{codeblock}

\tcode{rational} member functions:

\begin{itemdecl}
explicit rational(integer num);
\end{itemdecl}

\begin{itemdescr}
\effects Constructs a \tcode{rational} with a value of \tcode{num}.
\end{itemdescr}

\begin{itemdecl}
explicit rational(float val);
explicit rational(double val);
explicit rational(long double val);
\end{itemdecl}

\begin{itemdescr}
\effects Constructs a \tcode{rational} with a value equal to \tcode{val}.
\end{itemdescr}

\begin{itemdecl}
rational(integer num, integer den);
\end{itemdecl}

\begin{itemdescr}
\requires \tcode{den != 0}
\effects Constructs a \tcode{rational} given the specified numerator and denominator.
\end{itemdescr}

\begin{itemdecl}
rational& operator=(integer num);
\end{itemdecl}

\begin{itemdescr}
\effects Assigns \tcode{num} to \tcode{*this}.

\returns \tcode{*this}.
\end{itemdescr}

\begin{itemdecl}
rational& assign(integer num, integer den);
\end{itemdecl}

\begin{itemdescr}
\requires \tcode{den != 0}

\effects Assigns the specified numerator and denominator to \tcode{*this}.

\returns \tcode{*this}.
\end{itemdescr}

\begin{itemdecl}
void swap(rational& rhs) noexcept;
\end{itemdecl}

\begin{itemdescr}
\effects Swaps \tcode{*this} and \tcode{rhs}.
\end{itemdescr}

\begin{itemdecl}
rational normalize() const;
\end{itemdecl}

\begin{itemdescr}
\returns A \tcode{rational} equal to \tcode{*this}, but with the numerator and denominator having no common factor other than 1 and the denominator greater than 0. If the numerator is 0, the denominator shall be 1.
\end{itemdescr}

\begin{itemdecl}
integer numer() const;
\end{itemdecl}

\begin{itemdescr}
\returns The (possibly not normalized) numerator by value.
\end{itemdescr}

\begin{itemdecl}
integer denom() const;
\end{itemdecl}

\begin{itemdescr}
\returns The (possibly not normalized) denominator by value.
\end{itemdescr}

\begin{itemdecl}
explicit operator bool() const noexcept;
\end{itemdecl}

\begin{itemdescr}
\returns As if \tcode{*this != 0}.
\end{itemdescr}

\begin{itemdecl}
rational& negate() noexcept;
\end{itemdecl}

\begin{itemdescr}
\effects Changes the sign of \tcode{*this}.

\returns \tcode{*this}.
\end{itemdescr}

\begin{itemdecl}
rational& invert() noexept;
\end{itemdecl}

\begin{itemdescr}
\requires the numerator is non-zero.
    
\effects Swaps the numerator and denominator.

\returns \tcode{*this}.
\end{itemdescr}

\rSec2[rational.member.ops]{\tcode{rational } member operators:}

\begin{itemdecl}
rational& operator++();
\end{itemdecl}

\begin{itemdescr}
\effects Adds 1 to \tcode{*this} and stores the result in \tcode{*this}.

\returns \tcode{*this}.
\end{itemdescr}

\begin{itemdecl}
rational& operator--();
\end{itemdecl}

\begin{itemdescr}
\effects Subtracts 1 from \tcode{*this} and stores the result in \tcode{*this}.

\returns \tcode{*this}.
\end{itemdescr}

\begin{itemdecl}
rational operator++(int);
\end{itemdecl}

\begin{itemdescr}
\effects Adds 1 to \tcode{*this} and stores the result in \tcode{*this}.
\returns The value of \tcode{*this} before the addition.
\end{itemdescr}

\begin{itemdecl}
rational operator--(int);
\end{itemdecl}

\begin{itemdescr}
\effects Subtracts 1 from \tcode{*this} and stores the result in \tcode{*this}.

\returns The value of \tcode{*this} before the subtraction.
\end{itemdescr}

\begin{itemdecl}
rational& operator+=(const integer& rhs);
\end{itemdecl}

\begin{itemdescr}
\effects Adds the integer value \tcode{rhs} to \tcode{*this} and stores the result in \tcode{*this}.

\returns \tcode{*this}.
\end{itemdescr}

\begin{itemdecl}
rational& operator-=(const integer& rhs);
\end{itemdecl}

\begin{itemdescr}
\effects Subtracts the integer value \tcode{rhs} from \tcode{*this} and stores the result in \tcode{*this}.

\returns \tcode{*this}.
\end{itemdescr}

\begin{itemdecl}
rational& operator*=(const integer& rhs);
\end{itemdecl}

\begin{itemdescr}
\effects Multiplies \tcode{*this} by the integer value \tcode{rhs} and stores the result in \tcode{*this}.

\returns \tcode{*this}.
\end{itemdescr}

\begin{itemdecl}
rational& operator/=(const integer& rhs);
\end{itemdecl}

\begin{itemdescr}
\requires \tcode{rhs != 0}.
    
\effects Divides \tcode{*this} by the integer value \tcode{rhs} and stores the result in \tcode{*this}.

\returns \tcode{*this}.
\end{itemdescr}

\begin{itemdecl}
rational& operator+=(const rational& rhs);
\end{itemdecl}

\begin{itemdescr}
\effects Adds the rational value \tcode{rhs} to \tcode{*this} and stores the result in \tcode{*this}.

\returns \tcode{*this}.
\end{itemdescr}

\begin{itemdecl}
rational& operator-=(const rational& rhs);
\end{itemdecl}

\begin{itemdescr}
\effects Subtracts the rational value \tcode{rhs} from \tcode{*this} and stores the result in \tcode{*this}.

\returns \tcode{*this}.
\end{itemdescr}

\begin{itemdecl}
rational& operator*=(const rational& rhs);
\end{itemdecl}

\begin{itemdescr}
\effects Multiplies \tcode{*this} by the rational value \tcode{rhs} and stores the result in \tcode{*this}.

\returns \tcode{*this}.
\end{itemdescr}

\begin{itemdecl}
rational& operator/=(const rational& rhs);
\end{itemdecl}

\begin{itemdescr}
\requires \tcode{rhs != 0}.

\effects Divides \tcode{*this} by the rational value \tcode{rhs} and stores the result in \tcode{*this}.

\returns \tcode{*this}.
\end{itemdescr}

\rSec2[rational.ops]{\tcode{rational } non-member operators:}

\begin{itemdecl}
rational operator+(const rational& val);
\end{itemdecl}

\begin{itemdescr}
\returns \tcode{rational(val)}.
\end{itemdescr}

\begin{itemdecl}
rational operator-(const rational& val);
\end{itemdecl}

\begin{itemdescr}
\returns \tcode{rational(val).negate()}.
\end{itemdescr}

\begin{itemdecl}
rational operator+(const rational& lhs, const rational& rhs);
\end{itemdecl}

\begin{itemdescr}
\returns \tcode{rational(lhs) += rhs}.
\end{itemdescr}

\begin{itemdecl}
rational operator-(const rational& lhs, const rational& rhs);
\end{itemdecl}

\begin{itemdescr}
\returns \tcode{rational(lhs) -= rhs}.
\end{itemdescr}

\begin{itemdecl}
rational operator*(const rational& lhs, const rational& rhs);
\end{itemdecl}

\begin{itemdescr}
\returns \tcode{rational(lhs) *= rhs}.
\end{itemdescr}

\begin{itemdecl}
rational operator/(const rational& lhs, const rational& rhs);
\end{itemdecl}

\begin{itemdescr}
\requires \tcode{rhs != 0}.

\returns \tcode{rational(lhs) /= rhs}.
\end{itemdescr}

\begin{itemdecl}
rational operator+(const rational& lhs, const integer& rhs);
\end{itemdecl}

\begin{itemdescr}
\returns \tcode{rational(lhs) += rhs}.
\end{itemdescr}

\begin{itemdecl}
rational operator-(const rational& lhs, const integer& rhs);
\end{itemdecl}

\begin{itemdescr}
\returns \tcode{rational(lhs) -= rhs}.
\end{itemdescr}

\begin{itemdecl}
rational operator*(const rational& lhs, const integer& rhs);
\end{itemdecl}

\begin{itemdescr}
\returns \tcode{rational(lhs) *= rhs}.
\end{itemdescr}

\begin{itemdecl}
rational operator/(const rational& lhs, const integer& rhs);
\end{itemdecl}

\begin{itemdescr}
\requires \tcode{rhs != 0}.

\returns \tcode{rational(lhs) /= rhs}.
\end{itemdescr}

\begin{itemdecl}
rational operator+(const integer& lhs, const rational& rhs);
\end{itemdecl}

\begin{itemdescr}
\returns \tcode{rational(rhs) += lhs}.
\end{itemdescr}

\begin{itemdecl}
rational operator-(const integer& lhs, const rational& rhs);
\end{itemdecl}

\begin{itemdescr}
\returns \tcode{rational(rhs).negate() += lhs}.
\end{itemdescr}

\begin{itemdecl}
rational operator*(const integer& lhs, const rational& rhs);
\end{itemdecl}

\begin{itemdescr}
\returns \tcode{rational(rhs) *= lhs}.
\end{itemdescr}

\begin{itemdecl}
rational operator/(const integer& lhs, const rational& rhs);
\end{itemdecl}

\begin{itemdescr}
\requires \tcode{rhs != 0}.

\returns \tcode{rational(rhs).invert() *= lhs}.
\end{itemdescr}

\begin{itemdecl}
bool operator==(const rational& lhs, const rational& rhs);
\end{itemdecl}

\begin{itemdescr}
\returns As if \tcode{lhs.numer() * rhs.denom() == rhs.numer() * lhs.denom()}.
\end{itemdescr}

\begin{itemdecl}
strong_ordering operator<=>(const rational& lhs, const rational& rhs);
\end{itemdecl}

\begin{itemdescr}
\returns As if \tcode{lhs.numer() * rhs.denom() <=> rhs.numer() * lhs.denom()}.
\end{itemdescr}


\begin{itemdecl}
bool operator==(const rational& lhs, const integer& rhs);
\end{itemdecl}

\begin{itemdescr}
\returns As if after normalization \tcode{lhs.numer() == rhs \&\& lhs.denom() == 1}.
\end{itemdescr}

\begin{itemdecl}
strong_ordering operator<=>(const rational& lhs, const integer& rhs);
\end{itemdecl}

\begin{itemdescr}
\returns As if \tcode{lhs.numer() <=> rhs * lhs.denom()}.
\end{itemdescr}

\end{addedblock}
