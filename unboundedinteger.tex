%!TEX root = ts.tex
\begin{addedblock}
\rSec0[unbounded_types]{Unbounded Types}

\rSec1[numeric.integer_data_proxy.syn]{Class \tcode{integer_data_proxy}}

\begin{codeblock}
namespace std {
  class integer_data_proxy {

    // type names
    typedef unspecified data_type;
    typedef unspecified arithmetic_type;
    typedef unspecified uarithmetic_type;
    typedef unspecified iterator;
    typedef unspecified const_iterator;
    typedef unspecified reverse_iterator;
    typedef unspecified const_reverse_iterator;

    // constructors
    integer_data_proxy(const integer_data_proxy& rhs) = delete;
    integer_data_proxy(integer_data_proxy&& rhs);

    // assign
    integer_data_proxy& operator=(const integer_data_proxy& rhs) = delete;
    integer_data_proxy& operator=(integer_data_proxy&& rhs) = delete;

    // iterators
    iterator begin() noexcept;
    iterator end() noexcept;
    reverse_iterator rbegin() noexcept;
    reverse_iterator rend() noexcept;
    const_iterator cbegin() const noexcept;
    const_iterator cend() const noexcept;
    const_reverse_iterator crbegin() const noexcept;
    const_reverse_iterator crend() const noexcept;

    // element access
    data_type operator[](size_t pos) const;
    data_type& operator[](size_t pos);

    // capacity
    size_t size() const noexcept;
    size_t capacity() const noexcept;
    void reserve(size_t digits);
    void shrink_to_fit();
  };
} // namespace std
\end{codeblock}

The class describes an object that can be used to examine and modify the internal representation of an object of type \tcode{integer}. This allows advanced users to portably implement algorithms that are not provided natively.

There can be only one \tcode{integer_data_proxy} object associated with a particular \tcode{integer} object at any given time; that object is obtained by calling the \tcode{get_data_proxy} member function on the \tcode{integer} object. The resulting object can be moved but not copied.

\begin{itemdecl}
typedef unspecified arithmetic_type;
\end{itemdecl}

\begin{itemdescr}
The typedef defines a synonym for a signed arithmetic type that is large enough to hold the product of the largest values that the implementation will store in an object of type data_type.
\end{itemdescr}

\begin{itemdecl}
iterator begin();
\end{itemdecl}

\begin{itemdescr}
\returns An iterator object such that the iterators \tcode{[begin(), end())} point to the internal data elements of the \tcode{integer} object.
\end{itemdescr}

\begin{itemdecl}
size_t capacity() const noexcept;
\end{itemdecl}

\begin{itemdescr}
\returns The number of decimal digits that the \tcode{integer} object can represent without reallocating its internal storage.
\end{itemdescr}

\begin{itemdecl}
const_iterator cbegin() const;
\end{itemdecl}

\begin{itemdescr}
\returns An iterator object such that the iterator range \tcode{[cbegin(), cend())} points to the internal data elements of the \tcode{integer} object.
\end{itemdescr}

\begin{itemdecl}
const_iterator cend() const;
\end{itemdecl}

\begin{itemdescr}
\returns An iterator object such that the iterator range \tcode{[cbegin(), cend())} points to the internal data elements of the \tcode{integer} object.
\end{itemdescr}

\begin{itemdecl}
typedef unspecified const_iterator;
\end{itemdecl}

\begin{itemdescr}
The typedef defines a synonym for an iterator that can be used to access but not modify internal data elements of the \tcode{integer} object.
\end{itemdescr}

\begin{itemdecl}
typedef unspecified const_reverse_iterator;
\end{itemdecl}

\begin{itemdescr}
The typedef defines a synonym for a reverse iterator that can be used to access but not modify internal data elements of the \tcode{integer} object.
\end{itemdescr}

\begin{itemdecl}
const_reverse_iterator crbegin() const;
\end{itemdecl}

\begin{itemdescr}
\returns The member function returns a reverse iterator object such that the iterator range \tcode{[crbegin(), crend())} points to the internal data elements of the \tcode{integer} object in reverse order.
\end{itemdescr}

\begin{itemdecl}
const_reverse_iterator crend() const;
\end{itemdecl}

\begin{itemdescr}
\returns A reverse iterator object such that the iterator range \tcode{[crbegin(), crend())} points to the internal data elements of the \tcode{integer} object in reverse order.
\end{itemdescr}

\begin{itemdecl}
typedef unspecified data_type;
\end{itemdecl}

\begin{itemdescr}
The typedef defines a synonym for the type of the \tcode{integer} object's internal data elements.
\end{itemdescr}

\begin{itemdecl}
iterator end();
\end{itemdecl}

\begin{itemdescr}
\returns An iterator object such that the iterator range \tcode{[begin(), end())} points to the internal data elements of the \tcode{integer} object.
\end{itemdescr}

\begin{itemdecl}
integer_data_proxy(const integer_data_proxy&) = delete;
\end{itemdecl}

\begin{itemdescr}
The copy constructor is deleted.
\end{itemdescr}

\begin{itemdecl}
integer_data_proxy(integer_data_proxy&& rhs);
\end{itemdecl}

\begin{itemdescr}
\effects Copies the contents of \tcode{rhs} and leaves \tcode{rhs} in an unspecified valid state.
\end{itemdescr}

\begin{itemdecl}
typedef unspecified iterator;
\end{itemdecl}

\begin{itemdescr}
The typedef defines a synonym for an iterator that can be used to access internal data elements of the \tcode{integer} object.
\end{itemdescr}

\begin{itemdecl}
integer& operator=(const integer_data_proxy&) = delete;
integer& operator=(integer_data_proxy&&) = delete;
\end{itemdecl}

\begin{itemdescr}
The copy assignment and move assignment operators are deleted.
\end{itemdescr}

\begin{itemdecl}
data_type operator[](size_t pos) const;
\end{itemdecl}

\begin{itemdescr}
\returns The value of the internal data element at index \tcode{pos}.
\end{itemdescr}

\begin{itemdecl}
data_type& operator[](size_t pos);
\end{itemdecl}

\begin{itemdescr}
\returns A reference to the internal data element at index \tcode{pos}.
\end{itemdescr}

\begin{itemdecl}
reverse_iterator rbegin();
\end{itemdecl}

\begin{itemdescr}
\returns A reverse iterator object such that the iterator range \tcode{[crbegin(), crend())} points to the internal data elements of the \tcode{integer} object in reverse order.
\end{itemdescr}

\begin{itemdecl}
reverse_iterator rend();
\end{itemdecl}

\begin{itemdescr}
\returns A reverse iterator object such that the iterator range \tcode{[crbegin(), crend())} points to the internal data elements of the \tcode{integer} object in reverse order.
\end{itemdescr}

\begin{itemdecl}
void reserve(size_t digits);
\end{itemdecl}

\begin{itemdescr}
\effects Ensures that \tcode{capacity() >= digits}.
\end{itemdescr}

\begin{itemdecl}
typedef unspecified reverse_iterator;
\end{itemdecl}

\begin{itemdescr}
The typedef defines a synonym for a reverse iterator that can be used to access internal data elements of the \tcode{integer} object.
\end{itemdescr}

\begin{itemdecl}
void shrink_to_fit();
\end{itemdecl}

\begin{itemdescr}
\effect Is a non-binding request to reduce \tcode{capacity()} to hold the \tcode{integer} object's current stored value without wasted space.
\end{itemdescr}

\begin{itemdecl}
size_t size() const;
\end{itemdecl}

\begin{itemdescr}
\returns \tcode{capacity()}.
\end{itemdescr}

\begin{itemdecl}
typedef unspecified uarithmetic_type;
\end{itemdecl}

\begin{itemdescr}
The typedef defines a synonym for an unsigned arithmetic type that is large enough to hold the product of the largest values that the implementation will store in an object of type data_type.
\end{itemdescr}

\indexlibrary{\idxhdr{Bits}}%
\rSec1[numeric.bits.syn]{Bits}

\begin{codeblock}
namespace std {

  // 26.??.?? binary operations
  bits operator&(const bits& lhs, const bits& rhs);
  bits operator|(const bits& lhs, const bits& rhs);
  bits operator^(const bits& lhs, const bits& rhs);

  // 26.??.?? iostream specializations
  template<class CharT, class Traits>
    basic_ostream<CharT, Traits>& operator<<(basic_ostream<CharT, Traits>& os,
                                             const bits& val);

  template<class CharT, class Traits>
    basic_ostream<CharT, Traits>& operator>>(basic_ostream<CharT, Traits>& os,
                                             bits& val);

} // namespace std
\end{codeblock}

\begin{codeblock}
namespace std {
  class bits {
  public:

    class reference;

    // constructors
    bits() noexcept;

    template <class Ty>
      bits(Ty rhs) noexcept;    // integral types only

    bits(initializer_list<uint_least32_t> list);

    template <class CharT, class Traits, class Alloc>
      explicit bits(const basic_string<CharT, Traits, Alloc>& str,
                    typename basic_string<CharT, Traits, Alloc>::size_t pos = 0,
                    typename basic_string<CharT, Traits, Alloc>::size_t count = std::basic_string<CharT>::npos,
                    CharT zero = CharT('0'),
                    CharT one = CharT('1'));

  template <class CharT>
      explicit bits(const CharT *ptr,
                  typename basic_string<CharT>::size_t count = std::basic_string<CharT>::npos,
                  CharT zero = CharT('0'),
                   CharT one = CharT('1'));

  explicit bits(const integer& val);
  explicit bits(integer&& val);

  bits(const bits& rhs);
  bits(bits&& rhs) noexcept;

    // assign and swap
    template <class Ty>
      bits& operator=(Ty rhs); // integral types only
    bits& operator=(const integer& rhs);
    bits& operator=(integer&& rhs);
    bits& operator=(const bits& rhs);
    bits& operator=(bits&& rhs);
    void swap(bits& rhs) noexcept;

    // conversions
    unsigned long to_ulong() const;
    unsigned long long to_ullong() const;
    template <class CharT = char, class Traits = char_traits<CharT>, class Alloc = allocator<CharT>>
      basic_string<CharT, Traits, Alloc> to_string(CharT zero = CharT('0'), CharT one = CharT('1')) const;

    // logical operations
    bits& operator&=(const bits& rhs);
    bits& operator|=(const bits& rhs);
    bits& operator^=(const bits& rhs);
    bits operator~() const;

    bits& operator<<=(size_t rhs);
    bits& operator>>=(size_t rhs);
    bits& operator<<(size_t rhs) const;
    bits& operator>>(size_t rhs) const;

    // element access and modification
    bits& set() noexcept;
    bits& set(size_t pos, bool val = true);
    bits& reset() noexcept;
    bits& reset(size_t pos);
    bits& flip() noexcept;
    bits& flip(size_t pos);
    bool operator[](size_t pos) const;
    reference operator[](size_t pos);
    bool test(size_t pos) const noexcept;
    bool all() const noexcept;
    bool any() const noexcept;
    bool none() const noexcept;
    size_t count() const noexcept;
    size_t count_not_set() const noexcept;

    // comparison
    bool operator==(const bits& rhs) const noexcept;
    bool operator!=(const bits& rhs) const noexcept;

    // capacity
    size_t size() const noexcept;
    size_t capacity() const noexcept;
    void reserve(size_t bit_count);
    void shrink_to_fit();
  };
} // namespace std
\end{codeblock}

The class describes an object that represents an unbounded set of bits.

\rSec2[numeric.bits.cons]{Constructors}

\begin{itemdecl}
bits() noexcept;
\end{itemdecl}

\begin{itemdescr}
\effects Constructs an object whose value is 0.
\end{itemdescr}

\begin{itemdecl}
template <class Ty>
  bits(Ty rhs) noexcept;    // integral types only
\end{itemdecl}

\begin{itemdescr}
\effects Constructs an object whose value is the ones-complement representation of \tcode{rhs}. Shall not take part in overload resolution unless the type \tcode{Ty} is an integral type.
\end{itemdescr}

\begin{itemdecl}
bits(initializer_list<uint_least32_t> list);
\end{itemdecl}

\begin{itemdescr}
\effects Constructs an object whose stored value is equal to the elements of the \tcode{initializer_list} treated as a series of unsigned 32-bit digits with the leftmost digit being most significant. For example, the \tcode{initializer_list { 0xFE, 0xF0, 0xAA, 0x31 }} represents the value \tcode{0xFE * 323 + 0xF0 * 322 + 0xAA * 321 + 0x31 * 320}.
\end{itemdescr}

\begin{itemdecl}
template <class CharT, class Traits, class Alloc>
  explicit bits(const basic_string<CharT, Traits, Alloc>& str,
                typename basic_string<CharT, Traits, Alloc>::size_t pos = 0,
                typename basic_string<CharT, Traits, Alloc>::size_t count = basic_string<CharT>::npos,
                CharT zero = CharT('0'),
                CharT one = CharT('1'));
template <class CharT>
  explicit bits(const CharT *ptr,
                typename basic_string<CharT>::size_t count = basic_string<CharT>::npos,
                CharT zero = CharT('0'),
                CharT one = CharT('1'));
\end{itemdecl}

\begin{itemdescr}
\effects Construct an object whose value is the value represented by their argument, treating zero as 0 and one as 1.
\end{itemdescr}

\begin{itemdecl}
explicit bits(const integer& rhs);
explicit bits(integer&& rhs);
\end{itemdecl}

\begin{itemdescr}
\effects Construct objects whose value is the ones-complement representation of \tcode{rhs}.
\end{itemdescr}

\begin{itemdecl}
bits(const bits& rhs);
bits(bits&& rhs) noexcept;
\end{itemdecl}

\begin{itemdescr}
\effects Construct objects with the same value as \tcode{rhs}. The move constructor leaves \tcode{rhs} in an unspecified valid state.
\end{itemdescr}

\rSec2[numeric.bits.ops]{Operations}

\begin{itemdecl}
size_t capacity() const noexcept;
\end{itemdecl}

\begin{itemdescr}
\returns The number of bits that the object can represent without reallocating its internal storage.
\end{itemdescr}

\begin{itemdecl}
size_t count() const noexcept;
\end{itemdecl}

\begin{itemdescr}
\returns The number of bits in \tcode{*this} that are set, or \tcode{static_cast<size_t>(-1)} if the number of bits that are set is too large to fit in an object of type \tcode{size_t}.
\end{itemdescr}

\begin{itemdecl}
size_t count_not_set() const noexcept;
\end{itemdecl}

\begin{itemdescr}
\returns The number of bits in \tcode{*this} that are not set, or \tcode{static_cast<size_t>(-1)} if the number of bits that are not set is too large to fit in an object of type \tcode{size_t}.
\end{itemdescr}

\begin{itemdecl}
void flip() const noexcept;
\end{itemdecl}

\begin{itemdescr}
\effects Toggles all the bits in the stored value.
\end{itemdescr}

\begin{itemdecl}
void flip(size_t pos);
\end{itemdecl}

\begin{itemdescr}
\effects Toggles the bit at position \tcode{pos} in the stored value.
\end{itemdescr}

\begin{itemdecl}
bool none() const noexcept;
\end{itemdecl}

\begin{itemdescr}
\returns True only if none of the bits in \tcode{*this} is set.
\end{itemdescr}

\begin{itemdecl}
void reserve(size_t bit_count);
\end{itemdecl}

\begin{itemdescr}
\effects Ensures that \tcode{capacity() >= bit_count}.
\end{itemdescr}

\begin{itemdecl}
bits& reset() noexcept;
\end{itemdecl}

\begin{itemdescr}
\effects Clears all the bits of \tcode{*this}.

\returns \tcode{*this}.
\end{itemdescr}

\begin{itemdecl}
bits& reset(size_t pos);
\end{itemdecl}

\begin{itemdescr}
\effects Clears the bit as position \tcode{pos}.

\returns \tcode{*this}.
\end{itemdescr}

\begin{itemdecl}
void set() noexcept;
\end{itemdecl}

\begin{itemdescr}
\effects Sets all the bits of \tcode{*this}.

\returns {*this}.
\end{itemdescr}

\begin{itemdecl}
void set(size_t pos, bool val = true);
\end{itemdecl}

\begin{itemdescr}
\effects Sets the bit at position \tcode{pos} in the stored value to \tcode{val}.

\returns \tcode{*this}.
\end{itemdescr}

\begin{itemdecl}
void shrink_to_fit();
\end{itemdecl}

\begin{itemdescr}
\effects Is a non-binding request to reduce \tcode{capacity()} to hold the current stored value without wasted space.
\end{itemdescr}

\begin{itemdecl}
size_t size() const noexcept;
\end{itemdecl}

\begin{itemdescr}
\returns \tcode{capacity()}.
\end{itemdescr}

\begin{itemdecl}
bool test(size_t pos) const noexcept;
\end{itemdecl}

\begin{itemdescr}
\returns True only if the bit at position \tcode{pos} in the stored value is non-zero.
\end{itemdescr}

\rSec2[numeric.bits.conv]{Bits conversion}

\begin{itemdecl}
template <class CharT = char, class Traits = char_traits<CharT>, class Alloc = allocator<CharT>>
  basic_string<CharT, Traits, Alloc> to_string(CharT zero = CharT('0'),
                                               CharT one = CharT('1'));
\end{itemdecl}

\begin{itemdescr}
\returns A string representation of the bits in the value stored in \tcode{*this}, using zero to represent 0 and one to represent 1.
\end{itemdescr}

\begin{itemdecl}
unsigned long long to_ullong() const;
\end{itemdecl}

\begin{itemdescr}
\returns A value equal to the stored value of \tcode{*this}. It throws an exception of type \tcode{range_error} if the value cannot be represented as an \tcode{unsigned long long}.
\end{itemdescr}

\begin{itemdecl}
unsigned long to_ulong() const;
\end{itemdecl}

\begin{itemdescr}
\returns A value equal to the stored value of \tcode{*this}. It throws an exception of type \tcode{range_error} if the value cannot be represented as a \tcode{long long}.
\end{itemdescr}

\rSec2[numeric.bits.operators]{Bits operators}

\begin{itemdecl}
template <class Ty>
  bits& operator=(Ty rhs);   // integral types only
\end{itemdecl}

\begin{itemdescr}
\effects Shall not take part in overload resolution unless the type \tcode{Ty} is an arithmetic type. The operator effectively executes \tcode{*this = integer(rhs)}.

\returns \tcode{*this}.
\end{itemdescr}

\begin{itemdecl}
bits& operator=(const bits& rhs);
bits& operator=(bits&& rhs);
\end{itemdecl}

\begin{itemdescr}
\effects Store the value of \tcode{rhs} into \tcode{*this}.

\returns \tcode{*this}.
\end{itemdescr}

\begin{itemdecl}
bits& operator=(const integer& rhs);
bits& operator=(integer&& rhs);
\end{itemdecl}

\begin{itemdescr}
\effects Store the ones-complement representation of \tcode{rhs} into \tcode{*this}.

\returns \tcode{*this}.
\end{itemdescr}

\begin{itemdecl}
bool operator==(const bits& rhs) const noexcept;
\end{itemdecl}

\begin{itemdescr}
\returns \tcode{True} only if the stored value in \tcode{*this} is the same as the stored value in \tcode{rhs}.
\end{itemdescr}

\begin{itemdecl}
bool operator!=(const bits& rhs) const noexcept;
\end{itemdecl}

\begin{itemdescr}
\returns \tcode{!(*this == rhs)}.
\end{itemdescr}

\begin{itemdecl}
bits operator&(const bits& lhs, const bits& rhs);
\end{itemdecl}

\begin{itemdescr}
\returns An object whose value is the bitwise AND of the values of \tcode{lhs} and \tcode{rhs}.
\end{itemdescr}

\begin{itemdecl}
bits& operator&=(const bits& rhs);
\end{itemdecl}

\begin{itemdescr}
\effects Sets the value of \tcode{*this} to the bitwise AND of the values of \tcode{*this} and \tcode{rhs}.

\returns A reference to \tcode{*this}.
\end{itemdescr}

\begin{itemdecl}
bits operator|(const bits& lhs, const bits& rhs);
\end{itemdecl}

\begin{itemdescr}
\returns An object whose value is the bitwise inclusive OR of the values of \tcode{lhs} and \tcode{rhs}.
\end{itemdescr}

\begin{itemdecl}
bits& operator|=(const bits& rhs);
\end{itemdecl}

\begin{itemdescr}
\effects Sets the value of \tcode{*this} to the bitwise inclusive OR of the values of \tcode{*this} and \tcode{rhs}.

\returns \tcode{*this}.
\end{itemdescr}

\begin{itemdecl}
bits operator^(const bits& lhs, const bits& rhs);
\end{itemdecl}

\begin{itemdescr}
\returns An object whose value is the bitwise exclusive OR of the values of \tcode{lhs} and \tcode{rhs}.
\end{itemdescr}

\begin{itemdecl}
bits& operator^=(const bits& rhs);
\end{itemdecl}

\begin{itemdescr}
\effects Sets the value of \tcode{*this} to the bitwise exclusive OR of the values of \tcode{*this} and \tcode{rhs}.

\returns \tcode{*this}.
\end{itemdescr}

\begin{itemdecl}
bits operator~() const;
\end{itemdecl}

\begin{itemdescr}
\returns An object that holds the complement of the set of bits held by \tcode{*this}.
\end{itemdescr}

\begin{itemdecl}
bits operator>>(const bits& lhs, size_t rhs);
\end{itemdecl}

\begin{itemdescr}
\returns An object whose stored value is the value of the bits in \tcode{lhs} shifted right \tcode{rhs} positions.
\end{itemdescr}

\begin{itemdecl}
bits& operator>>=(size_t rhs);
\end{itemdecl}

\begin{itemdescr}
\effects Sets the stored value in \tcode{*this} to the value of the bits in \tcode{*this} shifted right \tcode{rhs} positions.

\returns \tcode{*this}.
\end{itemdescr}

\begin{itemdecl}
template <class CharT, class Traits>
  basic_istream<CharT, Traits>& operator>>(basic_istream<CharT, Traits>& is,
                                           bits& val);
\end{itemdecl}

\begin{itemdescr}
\effects Has the effect of \tcode{{ string temp; is >> temp; val = temp; }}.

\returns \tcode{is}.
\end{itemdescr}

\begin{itemdecl}
bits operator<<(const bits& lhs, size_t rhs);
\end{itemdecl}

\begin{itemdescr}
\returns An object whose stored value is the value of the bits in \tcode{lhs} shifted left \tcode{rhs} positions.
\end{itemdescr}

\begin{itemdecl}
bits& operator<<=(size_t rhs);
\end{itemdecl}

\begin{itemdescr}
\effects Sets the stored value in \tcode{*this} to the value of the bits in \tcode{*this} shifted left \tcode{rhs} positions.

\returns \tcode{*this}.
\end{itemdescr}

\begin{itemdecl}
template <class CharT, class Traits>
  basic_ostream<CharT, Traits>& operator<<(basic_ostream<CharT, Traits>& os,
                                           const bits& val);
\end{itemdecl}

\begin{itemdescr}
\effects Has the effect of \tcode{os << val.to_string()}.

\returns \tcode{os}.
\end{itemdescr}

\begin{itemdecl}
bool operator[](size_t pos) const;
\end{itemdecl}

\begin{itemdescr}
\returns The value of the bit at position \tcode{pos}.
\end{itemdescr}

\begin{itemdecl}
reference operator[](size_t pos);
\end{itemdecl}

\begin{itemdescr}
\returns An object of type \tcode{bits::reference} that refers to the bit at position \tcode{pos}.
\end{itemdescr}

\rSec2[numeric.bits.reference]{bits::reference class}

\begin{codeblock}
namespace std {
  class bits {
    class reference {
    public:
      reference& operator=(bool val) noexcept;
      reference& operator=(const reference& rhs) noexcept;
      bool operator~() const noexcept;
      operator bool() const noexcept;
      reference& flip() noexcept;
    };
  };
} // namespace std
\end{codeblock}

The nested class \tcode{bits::reference} describes an object that can be used to manage a particular bit in an object of type \tcode{bits}.

\begin{itemdecl}
reference& flip() noexcept;
\end{itemdecl}

\begin{itemdescr}
\effects Toggles the bit that the object manages.
\end{itemdescr}

\begin{itemdecl}
reference& operator=(bool rhs) noexcept;
\end{itemdecl}

\begin{itemdescr}
\effects Sets the bit that the object manages to the value of \tcode{rhs}.
\end{itemdescr}

\begin{itemdecl}
reference& operator=(const reference& rhs) noexcept;
\end{itemdecl}

\begin{itemdescr}
\effects Sets the bit that the object manages to the value managed by \tcode{rhs}.
\end{itemdescr}

\begin{itemdecl}
bool operator~() const noexcept;
\end{itemdecl}

\begin{itemdescr}
\returns True if the bit managed by the object is set, otherwise false.
\end{itemdescr}

\begin{itemdecl}
operator bool() const noexcept;
\end{itemdecl}

\begin{itemdescr}
\returns True if the bit that the object manages is set.
\end{itemdescr}

\rSec1[numeric.integer.syn]{Integer}

\begin{codeblock}
#include <compare>

namespace std {
  class integer {
  public:
    // constructors
    constexpr integer() noexcept;

    template <class Ty>
      integer(Ty rhs) noexcept; // arithmetic types only

    integer(initializer_list<uint_least32_t> init);

    template <class CharT, class Traits, class Alloc>
      explicit integer(const basic_string<CharT, Traits, Alloc>& str);

    explicit integer(const bits& rhs);
    explicit integer(bits&& rhs);

    integer(const integer& rhs);
    integer(integer&& rhs) noexcept;

    // assign and swap
    template <class Ty>
    integer& operator=(Ty rhs);   // arithmetic types only
    integer& operator=(const bits& rhs);
    integer& operator=(bits&& rhs);
    integer& operator=(const integer& rhs);
    integer& operator=(integer&& rhs);
    void swap(integer& rhs) noexcept;

    // conversions
    explicit operator long long() const;
    explicit operator unsigned long long() const;
    explicit operator long double() const noexcept;
    explicit operator bool() const noexcept;

    // comparisons
    int compare(const integer& rhs) const noexcept;

    // arithmetic operations
    integer& operator+=(const integer& rhs);
    integer& operator-=(const integer& rhs);
    integer& operator*=(const integer& rhs);
    integer& operator/=(const integer& rhs);
    integer& operator%=(const integer& rhs);

    integer& operator++();
    integer operator++(int);
    integer& operator--();
    integer operator--(int);

    integer div(const integer& rhs);

    integer& abs() noexcept;
    integer& negate() noexcept;
    integer operator+() const noexcept;
    integer operator-() const noexcept;

    integer& operator<<=(size_t rhs);
    integer& operator>>=(size_t rhs);

    // numeric operations
    integer& sqr();
    integer& sqrt();
    integer& pow(const integer& exp);
    integer& mod(const integer& rhs);
    integer& mulmod(const integer& rhs, const integer& m);
    integer& powmod(const integer& exp, const integer& m);

    // observers
    bool is_zero() const noexcept;
    bool is_odd() const noexcept;

    // accessors
    integer_data_proxy get_data_proxy();

    // capacity
    size_t size() const noexcept;
    size_t capacity() const noexcept;
    void reserve(size_t digits);
    void shrink_to_fit();
  };
} // namespace std

void swap(integer& lhs, integer& rhs) noexcept;

// comparisons
bool operator==(const integer& lhs, const integer& rhs) noexcept;
strong_ordering operator<=>(const integer& lhs, const integer& rhs) noexcept;

// arithmetic operations
integer operator+(const integer& lhs, const integer& rhs);
integer operator-(const integer& lhs, const integer& rhs);
integer operator*(const integer& lhs, const integer& rhs);
integer operator/(const integer& lhs, const integer& rhs);
integer operator%(const integer& lhs, const integer& rhs);

pair<integer, integer> div(const integer& lhs, const integer& rhs);

integer abs(const integer& val);

integer operator<<(const integer& lhs, size_t rhs);
integer operator>>(const integer& lhs, size_t rhs);

// numeric operations
integer sqr(const integer& val);
integer sqrt(const integer& val);
integer pow(const integer& val, const integer& exp);
integer mod(const integer& lhs, const integer& rhs);
integer mulmod(const integer& lhs, const integer& rhs, const integer& m);
integer powmod(const integer& lhs, const integer& rhs, const integer& m);

integer gcd(const integer& a, const integer& b);
integer lcm(const integer& a, const integer& b);

// conversions
string to_string(const integer& val, int radix = 10);

// I/O operations
template <class CharT, class Traits>
  basic_ostream<CharT, Traits>& operator<<(basic_ostream<CharT, Traits>& str,
                                           const integer& val);
template <class CharT, class Traits>
  basic_istream<CharT, Traits>& operator>>(basic_istream<CharT, Traits>& str,
                                           integer& val);
\end{codeblock}

The class describes an object that manages an unbounded-precision signed integral type that can be used in most contexts where an int could be used.

Any function specified to return an object of type \tcode{integer} may return an object of another type, provided all the const member functions of the class \tcode{integer} are also applicable to that type.

\begin{itemdecl}
integer abs(const integer& other);
\end{itemdecl}

\begin{itemdescr}
\returns An object that holds the absolute value of \tcode{other}.
\end{itemdescr}

\begin{itemdecl}
integer& abs() noexcept;
\end{itemdecl}

\begin{itemdescr}
\effects Sets the stored value of *this to its absolute value and returns \tcode{*this}.
\end{itemdescr}

\begin{itemdecl}
size_t capacity() const noexcept;
\end{itemdecl}

\begin{itemdescr}
\returns The number of decimal digits that the object can represent without reallocating its internal storage.
\end{itemdescr}

\begin{itemdecl}
int compare(const integer& rhs) const noexcept;
\end{itemdecl}

\begin{itemdescr}
\returns A value less than 0 if \tcode{*this} is less than \tcode{rhs}, 0 if \tcode{*this} is equal to \tcode{rhs}, and greater than 0 if \tcode{*this} is greater than \tcode{rhs}.
\end{itemdescr}

\begin{itemdecl}
pair<integer, integer> div(const integer& lhs, const integer& rhs);
\end{itemdecl}

\begin{itemdescr}
\returns An object that is an instantiation of \tcode{pair}; its first field holds the quotient, \tcode{lhs / rhs}, and its second field holds the remainder, \tcode{lhs \% rhs}.
\end{itemdescr}

\begin{itemdecl}
integer div(const integer& rhs) const;
\end{itemdecl}

\begin{itemdescr}
\returns The remainder, \tcode{*this \% rhs}, and stores the quotient, \tcode{*this / rhs}, into \tcode{*this}.
\end{itemdescr}

\begin{itemdecl}
integer gcd(const integer& a, const integer& b);
\end{itemdecl}

\begin{itemdescr}
\returns An object whose value is the greatest common denominator of \tcode{a} and \tcode{b}.
\end{itemdescr}

\begin{itemdecl}
integer_data_proxy get_data_proxy();
\end{itemdecl}

\begin{itemdescr}
\returns An object of type \tcode{integer_data_proxy} that can be used to examine and modify the internal storage of \tcode{*this}. If an object of type \tcode{integer_data_proxy} that refers to \tcode{*this} exists at the time of a call to this function, the function throws an exception object of type \tcode{runtime_error}.
\end{itemdescr}

\rSec2[numeric.integer.ctors]{Constructors}

\begin{itemdecl}
constexpr integer() noexcept;
\end{itemdecl}

\begin{itemdescr}
\effects Constructs an object whose value is 0.
\end{itemdescr}

\begin{itemdecl}
template <class Ty>
  integer(Ty val) noexcept; // arithmetic types only
\end{itemdecl}

\begin{itemdescr}
\effects For integral types the constructor constructs an object whose value is \tcode{val}. For floating-point types the constructor constructs an object whose value is the value of \tcode{val} with any fractional part discarded. Shall not take part in overload resolution unless the type \tcode{Ty} is an arithmetic type.
\end{itemdescr}

\begin{itemdecl}
integer(initializer_list<unspecified> list);
\end{itemdecl}

\begin{itemdescr}
\effects Constructs an object whose stored value is equal to the elements of the \tcode{initializer_list} treated as a series of unsigned 32-bit digits with the leftmost digit being most significant. For example, the \tcode{initializer list { 0xFE, 0xF0, 0xAA, 0x31 }} represents the value \tcode{0xFE * 323 + 0xF0 * 322 + 0xAA * 321 + 0x31 * 320}.
\end{itemdescr}

\begin{itemdecl}
template<class CharT, class Traits, class Alloc>
  explicit integer(const basic_string<CharT, Traits, Alloc>& str);
\end{itemdecl}

\begin{itemdescr}
\effects Constructs an object whose value is the value represented by the \tcode{string} object. The \tcode{string} object shall have the form required for the \tcode{string} argument to the function \tcode{strtol} with a radix of base, and shall be interpreted as if by \tcode{strtol(str.c_str(), 0, base)}, except that the resulting value can never be outside the range of representable values.
\end{itemdescr}

\begin{itemdecl}
integer(const bits& rhs);
integer(bits&& rhs);
\end{itemdecl}

\begin{itemdescr}
\effects Construct an object whose stored value is the value in the bit pattern in \tcode{rhs} interpreted as a ones-complement representation of an integer value.
\end{itemdescr}

\begin{itemdecl}
integer(const integer& rhs);
integer(integer&& rhs) noexcept;
\end{itemdecl}

\begin{itemdescr}
\effects Construct objects with the same value as \tcode{rhs}. The move constructor leaves \tcode{rhs} in an unspecified valid state.
\end{itemdescr}

\rSec2[numeric.integer.ops]{Operations}

\begin{itemdecl}
bool is_odd() const noexcept;
\end{itemdecl}

\begin{itemdescr}
\returns True only if the stored value represents an odd number.
\end{itemdescr}

\begin{itemdecl}
bool is_zero() const noexcept;
\end{itemdecl}

\begin{itemdescr}
\returns True only if the stored value is zero.
\end{itemdescr}

\begin{itemdecl}
integer lcm(const integer& a, const integer& b);
\end{itemdecl}

\begin{itemdescr}
\returns An object whose value is the least common multiple of \tcode{a} and \tcode{b}.
\end{itemdescr}

\begin{itemdecl}
integer mod(const integer& lhs, const integer& rhs);
\end{itemdecl}

\begin{itemdescr}
\returns An object whose value is \tcode{lhs mod rhs}.
\end{itemdescr}

\begin{itemdecl}
integer& mod(const integer& rhs);
\end{itemdecl}

\begin{itemdescr}
\effects Sets the stored value in \tcode{*this} to \tcode{*this mod rhs}.

\returns \tcode{*this}.
\end{itemdescr}

\begin{itemdecl}
integer mulmod(const integer& lhs, const integer& rhs, const integer& m);
\end{itemdecl}

\begin{itemdescr}
\returns An object whose value is \tcode{(lhs * rhs) mod m}.
\end{itemdescr}

\begin{itemdecl}
integer& mulmod(const integer& rhs, const integer& m);
\end{itemdecl}

\begin{itemdescr}
\effects Sets the value of \tcode{*this} to \tcode{(*this * rhs) mod m}.

\returns \tcode{*this}.
\end{itemdescr}

\begin{itemdecl}
integer& negate() noexcept;
\end{itemdecl}

\begin{itemdescr}
\effects Sets the stored value of \tcode{*this} to the negation of its previous value.

\returns \tcode{this}.
\end{itemdescr}

\begin{itemdecl}
integer pow(const integer& val, const integer& exp);
\end{itemdecl}

\begin{itemdescr}
\requires \tcode{0 <= exp}.

\returns An object whose value is \tcode{$val^{exp}$}.
\end{itemdescr}

\begin{itemdecl}
integer& pow(const integer& exp);
\end{itemdecl}

\begin{itemdescr}
\requires \tcode{0 <= exp}.

\effects Sets the value of \tcode{*this} to \tcode{$*this^{exp}$}.

\returns \tcode{*this}.
\end{itemdescr}

\begin{itemdecl}
integer powmod(const integer& val, const integer& exp, const integer& m);
\end{itemdecl}

\begin{itemdescr}
\requires \tcode{0 <= exp} and \tcode{m != 0}.

\returns An object whose value is \tcode{$val^{exp}$ mod m}.
\end{itemdescr}

\begin{itemdecl}
integer& powmod(const integer& exp, const integer& m);
\end{itemdecl}

\begin{itemdescr}
\requires \tcode{0 <= exp} and \tcode{m != 0}.

\effects Sets the value of \tcode{*this} to \tcode{$*this^{exp}$ mod m}.

\returns \tcode{*this}.
\end{itemdescr}

\begin{itemdecl}
void reserve(size_t digits);
\end{itemdecl}

\begin{itemdescr}
\effects Ensures that \tcode{capacity() >= digits}.
\end{itemdescr}

\begin{itemdecl}
void shrink_to_fit();
\end{itemdecl}

\begin{itemdescr}
\effects A non-binding request to reduce \tcode{capacity()} to hold the current stored value without wasted space.
\end{itemdescr}

\begin{itemdecl}
size_t size() const noexcept;
\end{itemdecl}

\begin{itemdescr}
\returns \tcode{capacity()}.
\end{itemdescr}

\begin{itemdecl}
integer sqr(const integer& val);
\end{itemdecl}

\begin{itemdescr}
\returns An object whose value is \tcode{val * val}.
\end{itemdescr}

\begin{itemdecl}
integer& sqr();
\end{itemdecl}

\begin{itemdescr}
\effects Sets the value of \tcode{*this} to \tcode{*this * *this}.

\returns \tcode{*this}.
\end{itemdescr}

\begin{itemdecl}
integer sqrt(const integer& val);
\end{itemdecl}

\begin{itemdescr}
\requires \tcode{0 <= val}.

\returns An object whose value is the square root of the value held by \tcode{val}, discarding any fractional part.
\end{itemdescr}

\begin{itemdecl}
integer& sqrt();
\end{itemdecl}

\begin{itemdescr}
\requires \tcode{0 <= *this}.

\effects Sets the value of \tcode{*this} to the square root of the value held by \tcode{*this}, discarding any fractional part.

\returns \tcode{*this}.
\end{itemdescr}

\begin{itemdecl}
void swap(integer& lhs, integer& rhs) noexcept;
\end{itemdecl}

\begin{itemdescr}
\effects Swaps the stored values of \tcode{lhs} and \tcode{rhs}.
\end{itemdescr}

\begin{itemdecl}
void swap(integer& rhs) noexcept;
\end{itemdecl}

\begin{itemdescr}
\effects Swaps the stored values of \tcode{*this} and \tcode{rhs}.
\end{itemdescr}

\begin{itemdecl}
string to_string(const integer& val, int radix = 10) const;
\end{itemdecl}

\begin{itemdescr}
\returns A \tcode{string} representation of the value stored in \tcode{val}, using \tcode{radix} as the radix.
\end{itemdescr}

\rSec2[numeric.integer.conv]{Conversion}

\begin{itemdecl}
explicit operator bool() const noexcept;
\end{itemdecl}

\begin{itemdescr}
\returns \tcode{False} only if \tcode{*this} is equal to 0.
\end{itemdescr}

\begin{itemdecl}
explicit operator long double() const noexcept;
\end{itemdecl}

\begin{itemdescr}
\returns A value equal to the stored value of  \tcode{*this}. If the stored value is outside the range that can be represented by an object of type  \tcode{long double} the returned value is positive or negative infinity, as appropriate.
\end{itemdescr}

\begin{itemdecl}
explicit operator long long() const;
\end{itemdecl}

\begin{itemdescr}
\returns A value equal to the stored value of \tcode{*this}. If the stored value cannot be represented as a \tcode{long long} it throws an exception of type \tcode{range_error}.
\end{itemdescr}

\begin{itemdecl}
explicit operator unsigned long long() const;
\end{itemdecl}

\begin{itemdescr}
\returns A value equal to the stored value of \tcode{*this}. If the stored value cannot be represented as an \tcode{unsigned long long} it throws an exception of type \tcode{range_error}.
\end{itemdescr}

\rSec2[numeric.integer.comp]{Comparison}

\begin{itemdecl}
bool operator==(const integer& lhs, const integer& rhs) noexcept;
\end{itemdecl}

\begin{itemdescr}
\returns \tcode{True} only if the value stored in \tcode{lhs} is equal to the value stored in \tcode{rhs}.
\end{itemdescr}

\begin{itemdecl}
strong_ordering operator<=>(const integer& lhs, const integer& rhs) noexcept;
\end{itemdecl}

\begin{itemdescr}
\returns \tcode{strong_equality{lhs.compare(rhs)};}.
\end{itemdescr}

\rSec2[numeric.integer.assign]{Assignment}

\begin{itemdecl}
template <class Ty>
  integer& operator=(Ty rhs);   // arithmetic types only
\end{itemdecl}

\begin{itemdescr}
\effects Shall not take part in overload resolution unless the type \tcode{Ty} is an arithmetic type. The operator effectively executes \tcode{*this = integer(rhs)}.

\returns \tcode{*this}.
\end{itemdescr}

\begin{itemdecl}
integer& operator=(const integer& rhs);
integer& operator=(integer&& rhs);
\end{itemdecl}

\begin{itemdescr}
\effects Store the value of \tcode{rhs} into \tcode{*this}.

\returns \tcode{*this}.
\end{itemdescr}

\begin{itemdecl}
integer& operator=(const bits& rhs);
integer& operator=(bits&& rhs);
\end{itemdecl}

\begin{itemdescr}
\effects Store the value of \tcode{rhs}, interpreted as a ones-complement representation of an integer value, into \tcode{*this}.

\returns \tcode{*this}.
\end{itemdescr}

\rSec2[numeric.integer.arithmetic_ops]{Arithmetic operations}

\begin{itemdecl}
integer operator+(const integer& lhs, const integer& rhs);
\end{itemdecl}

\begin{itemdescr}
\returns An object whose value is the sum of the values of \tcode{lhs} and \tcode{rhs}.
\end{itemdescr}

\begin{itemdecl}
integer operator+() const noexcept;
\end{itemdecl}

\begin{itemdescr}
\returns A copy of \tcode{*this}.
\end{itemdescr}

\begin{itemdecl}
integer& operator+=(const integer& rhs);
\end{itemdecl}

\begin{itemdescr}
\effects Sets the stored value of \tcode{*this} to the sum of the values of \tcode{*this} and \tcode{rhs}.

\returns A reference to \tcode{*this}.
\end{itemdescr}

\begin{itemdecl}
integer& operator++();
\end{itemdecl}

\begin{itemdescr}
\effects Set the value stored in \tcode{*this} to \tcode{*this + 1}.

\returns \tcode{*this}.
\end{itemdescr}

\begin{itemdecl}
integer operator++(int);
\end{itemdecl}

\begin{itemdescr}
\returns An object whose value is the value stored in \tcode{*this} prior to the increment.
\end{itemdescr}

\begin{itemdecl}
integer operator-(const integer& lhs, const integer& rhs)
\end{itemdecl}

\begin{itemdescr}
\returns An object whose value is the difference between the values of \tcode{lhs} and \tcode{rhs}.
\end{itemdescr}

\begin{itemdecl}
integer operator-() noexcept;
\end{itemdecl}

\begin{itemdescr}
\returns An object whose value is the negation of the value of \tcode{*this}.
\end{itemdescr}

\begin{itemdecl}
integer& operator-=(const integer&);
\end{itemdecl}

\begin{itemdescr}
\effects Sets the stored value of \tcode{*this} to the difference between the values of \tcode{*this} and \tcode{rhs}.

\returns \tcode{*this}.
\end{itemdescr}

\begin{itemdecl}
integer& operator--();
\end{itemdecl}

\begin{itemdescr}
\effects Set the value stored in \tcode{*this} to \tcode{*this - 1}.

\returns \tcode{*this}.
\end{itemdescr}

\begin{itemdecl}
integer operator--(int);
\end{itemdecl}

\begin{itemdescr}
\effects Set the value stored in \tcode{*this} to \tcode{*this - 1}.

\returns An object whose value is the value stored in \tcode{*this} prior to the decrement.
\end{itemdescr}

\begin{itemdecl}
integer operator*(const integer& lhs, const integer& rhs);
\end{itemdecl}

\begin{itemdescr}
\returns An object whose value is the product of the values of \tcode{lhs} and \tcode{rhs}.
\end{itemdescr}

\begin{itemdecl}
integer& operator*=(const integer& rhs);
\end{itemdecl}

\begin{itemdescr}
\effects Sets the stored value of \tcode{*this} to the product of the values of \tcode{*this} and \tcode{rhs}.

\returns A reference to \tcode{*this}.
\end{itemdescr}

\begin{itemdecl}
integer operator/(const integer& lhs, const integer& rhs);
\end{itemdecl}

\begin{itemdescr}
\returns An object whose value is the quotient of the value of \tcode{lhs} divided by the value of \tcode{rhs}, discarding any fractional part.
\end{itemdescr}

\begin{itemdecl}
integer& operator/=(const integer& rhs);
\end{itemdecl}

\begin{itemdescr}
\effects Sets the stored value of \tcode{*this} to the quotient of the value of \tcode{*this} divided by the value of \tcode{rhs}, discarding any fractional part.

\returns \tcode{*this}.
\end{itemdescr}

\begin{itemdecl}
integer operator%(const integer&, const integer&);
\end{itemdecl}

\begin{itemdescr}
\returns An object whose value is the remainder of the value of \tcode{lhs} divided by the value of \tcode{rhs}. The remainder is the value such that \tcode{(lhs / rhs) * rhs + lhs \% rhs} is equal to \tcode{lhs}.
\end{itemdescr}

\begin{itemdecl}
integer& integer::operator%=(const integer&);
\end{itemdecl}

\begin{itemdescr}
\effects Sets the stored value of \tcode{*this} to the remainder of \tcode{*this} divided by the value of \tcode{rhs}.

\returns \tcode{*this}.
\end{itemdescr}

\begin{itemdecl}
integer operator>>(const integer& val, size_t rhs);
\end{itemdecl}

\begin{itemdescr}
\returns An object whose value is \tcode{val / $2^{rhs}$}.
\end{itemdescr}

\begin{itemdecl}
integer& operator>>=(size_t rhs);
\end{itemdecl}

\begin{itemdescr}
\effects Sets the value of \tcode{*this} to \tcode{*this / $2^{rhs}$}.

\returns \tcode{*this}.
\end{itemdescr}

\rSec2[numeric.integer.io]{I/O}

\begin{itemdecl}
template <class Elem, class Traits>
  basic_istream<Elem, Traits>& operator>>(basic_istream<Elem, Traits>& is, integer& val);
\end{itemdecl}

\begin{itemdescr}
\effects Has the effect of \tcode{{ string temp; is >> temp; val = integer(temp); }}.

\returns \tcode{is}.
\end{itemdescr}

\begin{itemdecl}
integer operator<<(const integer& val, size_t rhs);
\end{itemdecl}

\begin{itemdescr}
\returns An object whose value is \tcode{val * $2^{rhs}$}.
\end{itemdescr}

\begin{itemdecl}
integer& integer::operator<<=(size_t rhs);
\end{itemdecl}

\begin{itemdescr}
\effects Sets the value of \tcode{*this} to \tcode{*this * $2^{rhs}$}.

\returns \tcode{*this}.
\end{itemdescr}

\begin{itemdecl}
template <class Elem, class Traits>
  basic_ostream<Elem, Traits>& operator<<(basic_ostream<Elem, Traits>& os, const integer& val);
\end{itemdecl}

\begin{itemdescr}
\effects Has the effect of \tcode{os << to_string(val)}.

\returns \tcode{os}.
\end{itemdescr}
\end{addedblock}
